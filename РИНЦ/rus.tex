\documentclass[final,11pt,3p]{elsarticle}
\usepackage[utf8x]{inputenc}
\usepackage[T2A]{fontenc}
\usepackage[russian]{babel}
\usepackage{graphicx}
\usepackage{amssymb}
\usepackage{amsthm}

\newtheorem{theorem}{Theorem}
\newtheorem{lemma}{Lemma}
\newtheorem{definition}{Definition}
\newtheorem{remark}{Remark}

\usepackage{lineno}
\usepackage{longtable}
\usepackage{caption}

\makeatletter
\def\ps@pprintTitle{%
 \let\@oddhead\@empty
 \let\@evenhead\@empty
 \def\@oddfoot{}%
 \let\@evenfoot\@oddfoot}
\makeatother


\begin{document}

\begin{frontmatter}
\title{Исследование мощности нового критерия проверки равенства распределений двух выборок}

\author[label1]{В.Б. Мелас}
% \ead{vbmelas@yandex.ru}
\author[label1]{Д.И. Сальников}
% \ead{mejibkop.ru@mail.ru}
\address[label1]{Санкт-Петербургский государственный университет, \\
                 Санкт-Петербург, Россия}

\begin{abstract}
В статье представлен новый перестановочный критерий проверки равенства распределений двух выборок. С помощью стохастического моделирования показана высокая эффективность критерия. Для выборок из нормальных распределений и распределений Коши, отличающихся только сдвигом, достигаемая мощность критерия оказывается примерно равной мощности критериев Уилкоксона-Манна-Уитни и Колмогорова-Смирнова. Однако если распределения различаются параметром масштаба, мощность предлагаемого критерия значительно выше.
\end{abstract}

\begin{keyword}
Перестановочные критерии \sep критерии однородности \sep распределение Коши \sep нормальное распределение
\end{keyword}
\end{frontmatter}

% \pagestyle{empty}

\section{Постановка задачи}
\label{S:1}
Рассмотрим классическую задачу проверки гипотезы равенства распределений двух выборок
\begin{equation}
  \label{H0}
  H_0\,:\,F_1 = F_2
\end{equation}
против альтернативной гипотезы
\begin{equation}
  \label{H1}
  H_1\,:\,F_1 \not= F_2
\end{equation}
в случае двух независимых выборок $X=(X_{1},\ldots, X_{n})$ и $Y=(Y_{1},\ldots, Y_{m})$ с функциями распределения $F_1$ и $F_2$ соответственно.

Хорошо известно [see e.g. (Lehman,1986)], что в случае, когда оба распределения являются нормальными и различаются только параметром сдвига, классический критерий Стьюдента обладает некоторыми оптимальными свойствами. Если распределения не является нормальным, но все же отличаются только параметром сдвига, то вместо критерия Стьюдента часто используется широко известная U-статистика Вилкоксона-Манна-Уитни (WMW). Однако можно показать, что для выборок из нормальных распределений, отличающихся только параметром масштаба, мощность критерия WMW крайне мала. Если законы распределения выборок неизвестны, возможно применение некоторых универсальных методов, таких как критерии Колмогорова-Смирнова или Крамера-фон Мизеса (см. \cite{Buening2001}), но во многих случаях эти критерии могут обладать невысокой мощностью.

В работе \cite{AslanZech2005} был предложен критерий, основанный на U-статистике с логарифмическим ядром, и представлено исследование его мощности для одномерных и многомерных случаев в сравнении с несколькими альтернативными методами. Насколько известно авторам, нет аналитических результатов асимптотической мощности критерия. В данной работе представлен похожий, но отличный критерий, и приведены численные результаты его мощности в сравнении с широко известными критериями.

\section{The new test and its statistical motivation}

Assume that the distribution functions
$F_1$ and $F_2$ belongs to the class of distribution functions of random values  $\xi$, such that
\begin {equation}\label{Class}
E [\ln (1+ \xi^2)     ] < \infty.
\end{equation}
Many distributions and, in particular, the Caushy distribution have this property.
 \bigskip

Among all distributions with given parameters of shift and scale having this property the Caushy's one have the maximum entropy.\\


Consider the following test
\begin{equation}
 \Phi_{A}=\frac{1}{n(n-1)}\sum_{1\leq i<j\leq n} g(X_i-X_j),
\Phi_{B}=\frac{1}{m(m-1)}\sum_{1\leq i<j\leq m} g(Y_i-Y_j),
\end{equation}
\begin{equation}\label{K1}
\Phi_{AB}=-\frac{1}{nm}\sum_{i=1}^n\sum_{j=1}^m g(X_i-Y_j),
\Phi_{nm}=\Phi_{A}+ \Phi_{B}+ \Phi_{AB},
\end{equation}
where
$$
g(u)= -\ln (1+|u|^2)
$$
is under a constant term precision the logarithm of the density of the standard Caushy distribution.
(Note that Zech and Aslan (2005) took $g(u)=\ln(|u|)$).



We would like to have a test that is appropriate for two distributions that differ only by shift and scale parameters and belong to a rather general class of distributions.


In particular, we consider the class of distributions satisfying (3), but the approach can be generalized for other classes of distributions.

Consider the class of distributions given by the property (\ref{Class}). Note that if the parameters are known the test basing on likelihood ratio is the most powerful among tests with given parameters.


The test suggested above can be considered as an approximation of logarithm of this ratio for the Caushy distribution.





\section{The analytical study of asymptotic power}
\label{S:2}

Let us consider the case of two distributions having the property (\ref{Class}) and, in particular, the two that differ only by the shift parameters. To simplify notations assume that $m=n$. The case $m\ne n$ is similar.
Now the criterion  (4) - (5)  assumes the form
\begin{equation}
T_n=\Phi_{nn}= \frac{1}{n^2}\sum_{i,j=1}^n \ln(1 + (X_i - Y_j)^2)-\frac{1}{n(n-1)}\sum_{1\leq i<j\leq n}  \ln(1 + (X_i - X_j)^2)
\end{equation}
\begin{equation}
-  \frac{1}{n(n-1)}\sum_{1\leq i<j\leq n}  \ln(1 + (Y_i - Y_j)^2).
\end{equation}

Denote by $C(u,v)$ the Caushy distribution with the density function
$$
v/(\pi(v^2 + (x-u)^2)).
$$

 Let $f_1(x)$ denotes the density of $F_1$ and $f_2(x)$ denotes the density of $F_2$.
Denote
\begin{eqnarray*}
J_h =\int_R g(x-y-|h|/\sqrt{n})f_1(x)f_2(y)dxdy,
\end{eqnarray*}

If there exists the limit
\begin{eqnarray}
lim_{n\to \infty} n(J_h - J_0)
\end{eqnarray}
denote it by $J^*(h)$.

The basic analytical result of the present paper is the following
\begin{theorem} Consider the problem of testing hypothesis on the equality of two distributions (1)-(2) where both functions have the property (3).Then\\
(i) under the condition $n \to \infty$
the distribution function of $nT_n$  converges under $H_0$ to that of the random value
\begin{equation}\label{Distr}
(aZ + b)^2,
\end{equation}
where  $Z$ has the normal distribution with zero expectation and variance equal to 1, $a^2= J_0/3, b=0$.

(ii)
Let $F_1= F(\nu,\mu),F_2=F(\nu+\theta,\mu),$
where $\nu$  and $\mu$ be the shift and scale parameters of the distribution F, F is arbitrary distribution with property   (3) and
$\theta=h/\sqrt{n},h$ is an arbitrary given number. Then
the distribution function of $nT_n$  converges under $H_1$ to that of the random value
$$
(aZ + b)^2,
$$
where $a^2 = J_0/3, b=0$
for the case of $H_0$ and
$a^2= J_0/3, b^2 = J^*(h)$
for  $H_1$.
In this case the power of the criterion $T_n$ with significance $\alpha$ is asymptotically equal to that is given by the formula
$$
Pr\{Z\geq z_{1-\alpha/2}-\sqrt{\frac {3J^*(h)}{J_0}}\}
+ Pr\{Z\leq - z_{1-\alpha/2}-\sqrt{\frac {3J^*(h)}{J_0}}\}.
$$

(iii) If $F_1=C(\nu,1), F_2=C(\nu + \theta)$ then in the part (ii)
$a^2 =(2/3)\ln 3, b= h/3$ and
$$
Pr\{Z\geq z_{1-\alpha/2}-(1/\sqrt{6\ln 3})h\}
+ Pr\{Z\leq - z_{1-\alpha/2}-(1/\sqrt{6\
ln 3})h\}.
$$
\end{theorem}


The proof of the theorem is given in the Appendix.


\section{Simulation results}


 We found by a stochastic simulation that the formula presents an approximation of the power of the test $T_n$ with a good accuracy (see tables 1-3 below).

 At the tables 1-12 results for cases $n$ =100, 500, 1000 and different values of h with $\alpha=0.05$ are given for normal and Caushy distributions that differ either by shift or by scale parameters. The critical values were calculated in two way: by simulation of the initial distribution and by random permutations (we used 800 random permutation in all cases). It worth to be noted that the results are very similar. Since the permutation technics is more universal, it can be recommended for practical applications.


Note that in all these cases when the distributions differ only in the scale parameters the power of  $T_n$ and  that of the Wilcoxon-Mann-Whitney and the
Kolmogorov - Smirnov tests were approximately equal to each other.
It can pointed out also that if the variances are not standard  but are known we should simply make a the corresponding normalisation.
But for the cases where the distributions differ in scale parameters the Wilcoxon-Mann-Whitney is not appropriate at all and the power of  the Kolmogorov - Smirnov test is considerably lower.



\begin{longtable}{|c|c|c|c|c|c|c|c|c|c|c|}
  \caption{Power of tests for the Cauchy distribution, \\
           $X\sim C(0,1)$, $Y\sim C(h/\sqrt{n},1)$, \\
           samples size 500, 1000 iterations, 800 permutations in $T_n, perm$} \\
  \hline
  h & $T_n, perm$ & $T_n, sim$ & $formula$ & $wilcox.test$ & $ks.test$ \\ \hline
  1 & 5.8   & 6.1      & 6.8       & 6.4           & 6.4       \\
  2 & 11.6  & 11.6     & 12.2      & 12.6          & 13.9      \\
  3 & 21    & 21.8     & 21.5      & 22.2          & 24.3      \\
  5 & 50.9  & 51       & 49.5      & 48            & 57.9      \\
  7 & 82.2  & 82.4     & 77.8      & 75.6          & 85.9      \\
  9 & 96.2  & 96.5     & 93.9      & 93.2          & 97.2      \\ \hline
\end{longtable}

\begin{longtable}{|c|c|c|c|c|c|c|c|c|c|c|}
  \caption{Power of tests for the Cauchy distribution, \\
           $X\sim C(0,1)$, $Y\sim C(h/\sqrt{n},1)$, \\
           samples size 1000, 1000 iterations, 800 permutations in $T_n, perm$} \\
  \hline
  h  & $T_n, perm$ & $T_n, sim$ & $formula$ & $wilcox.test$ & $ks.test$ \\ \hline
  1  & 6.3   & 6        & 6.8       & 6.8           & 8.1       \\
  2  & 11.4  & 11.9     & 12.2      & 12.9          & 13.4      \\
  3  & 21    & 20.9     & 21.5      & 22.8          & 26.2      \\
  4  & 34.9  & 34.6     & 34.4      & 36.1          & 43        \\
  7  & 84    & 84.5     & 77.8      & 79.5          & 87.6      \\
  10 & 99    & 98.9     & 97.4      & 96.8          & 99.2      \\ \hline
\end{longtable}

\begin{longtable}{|c|c|c|c|c|}
  \caption{Power of tests for the Cauchy distribution, \\
           $X\sim C(0,1)$, $Y\sim C(0, 1 + h/\sqrt{n})$, \\
           samples size 100, 1000 iterations, 800 permutations in $T_n, perm$} \\
  \hline
  $h$ & $T_n, perm$ & $T_n, sim$ & $wilcox.test$ & $ks.test$ \\
  \hline
  % $h$ & $T_n, perm$ & $T_n, sim$ & $wilcox.test$ & $ks.test$ \\
  2 & 0.106 & 0.119 & 0.054 & 0.054 \\
  4 & 0.276 & 0.298 & 0.055 & 0.087 \\
  6 & 0.494 & 0.536 & 0.055 & 0.159 \\
  8 & 0.688 & 0.735 & 0.055 & 0.25 \\
  10 & 0.842 & 0.871 & 0.052 & 0.364 \\
  \hline
\end{longtable}

\begin{longtable}{|c|c|c|c|c|}
  \caption{Power of tests for the Cauchy distribution, \\
           $X\sim C(0,1)$, $Y\sim C(0, 1 + h/\sqrt{n})$, \\
           samples size 500, 1000 iterations, 800 permutations in $T_n, perm$} \\
  \hline
  $h$ & $T_n, perm$ & $T_n, sim$ & $wilcox.test$ & $ks.test$ \\
  \hline
  % $h$ & $T_n, perm$ & $T_n, sim$ & $wilcox.test$ & $ks.test$ \\
  2 & 0.094 & 0.1 & 0.045 & 0.063 \\
  4 & 0.285 & 0.306 & 0.048 & 0.14 \\
  6 & 0.545 & 0.565 & 0.05 & 0.261 \\
  8 & 0.795 & 0.805 & 0.052 & 0.433 \\
  10 & 0.93 & 0.94 & 0.052 & 0.622 \\
  \hline
\end{longtable}

\begin{longtable}{|c|c|c|c|c|}
  \caption{Power of tests for the Cauchy distribution, \\
           $X\sim C(0,1)$, $Y\sim C(0, 1 + h/\sqrt{n})$, \\
           samples size 1000, 1000 iterations, 800 permutations in $T_n, perm$} \\
  \hline
  $h$ & $T_n, perm$ & $T_n, sim$ & $wilcox.test$ & $ks.test$ \\
  \hline
  % $h$ & $T_n, perm$ & $T_n, sim$ & $wilcox.test$ & $ks.test$ \\
  2 & 0.102 & 0.105 & 0.05 & 0.076 \\
  4 & 0.324 & 0.338 & 0.052 & 0.138 \\
  6 & 0.611 & 0.628 & 0.052 & 0.279 \\
  8 & 0.848 & 0.856 & 0.052 & 0.474 \\
  10 & 0.961 & 0.971 & 0.054 & 0.679 \\
  \hline
\end{longtable}

\section{Conclusion}
In this paper we suggested a new test for equality of two distributions. Its asymptotic power was analytically established for the case of distributions that differ only by shift. For the case of Caushy distribution this formula was presented in a closed form  since it proved to be possible to calculate the corresponding integrals analytically.

By stochastic simulation we found that for the Normal and Caushy distributions that differ only by shift the power of the new test is approximately equal to that of the Wilcoxon-Mann-Whitney and the
Kolmogorov - Smirnov tests. However if the distributions differ by the scale parameter our simulations show that the new test is considerably better than the Kolmogorov - Smirnov test. And in this case the the Wilcoxon-Mann-Whitney test is not appropriate at all.

\section*{Acknowledgments}
The authors are indebted to professor
 Yakov Nikitin for the help in calculating the  integrals. Work of Viatcheslav Melas was supported by RFBR (grant N 20-01-00096).

\section{Appendix}
Proof of Theorem 1.

Let us begin with studying the asymptonic behaviour of the magnitude $E(nT_n)^2$.

\begin{lemma}
(1) If hypothesis $H_0$ is satisfied and $F_1$ posesses property (3) then there exists a finite  limit of $d=\lim E(nT_n)^2$ with $n \to \infty.$

(2)If hypothesis $H_1$ is satisfied for $F_2(x)= F_1(x-\theta,\theta= h/\sqrt{n}$ and $F_1$ and $F_2$ possess property (3)  then there exists a finite  limit of $E(nT_n)^2$ for $n \to \infty$ and it is given by the formula
$$
d + 2J^*(h)J_0 + J^*(h)^2.
$$
\end{lemma}
Proof of the lemma.\\
Note that $(nT_n)^2$ is equal to
\begin{eqnarray*}
n^2 [\frac{1}{n^2}\sum_{i,j=1}^n [g(X_i - Y_j)-J_0]-\frac{1}{ n(n-1)}\sum_{i<j,i,j=1}^n  [g(X_i - X_j)-J_0] - \\ \frac{1}{n(n-1)}[\sum_{i<j,i,j=1}^n  [g(Y_i - Y_j)-J_0]]^2,
\end{eqnarray*}
where $g(z)= \ln(1+z^2).$\\
The idea of the proof consists in the splitting the three squares of three sums/ including in this sum and three
pairwise products into peculiar sums of the identical structure.
Then to each peculiar sum either law of large numbers or central limit theorem is applied.\\

The square of the first sum,
\begin{eqnarray*}
n^2 \{\frac{1}{n^2}\sum_{i,j=1}^n [g(X_i - Y_j)-J_0]\}^2,
\end{eqnarray*}
can be represented by sum of the following peculiar sums.
\begin{eqnarray*}
1)\,\,n^2 (\frac{1}{n^2})^2 \sum_{i,j=1}^n [(g(X_i - Y_j) -J_1]^2,
\end{eqnarray*}
\begin{eqnarray*}
2a)\,\,n^2(\frac{1}{n^2})^2 \sum_{i,j=1,i \ne j}^n\sum_{k=1}^n [(g(X_i - Y_j)-J_0][(g(X_k - Y_j)-J_0],
\end{eqnarray*}
\begin{eqnarray*}
2b)\,\,n^2(\frac{1}{n^2})^2 \sum_{i,j=1,i \ne j}^n\sum_{k=1}^n [g(Y_k - X_i)-J_0][g (Y_k - X_j)-J_0],
\end{eqnarray*}
\begin{eqnarray*}
3)\,\, n^2 (\frac{1}{n^2})^2 \sum_{i,j=1}^n\sum_{{l,k=1}, (l \ne  k)or(i\ne j)}^n  [g (X_i - Y_j)-J_0][g(X_l - Y_k)^2)-J_0].
\end{eqnarray*}

Similar expression can be obtained for each of the two other squares and three  pairwise
products.
Note the limit with $n \to \infty$ for the peculiar sums of the tipe 1) is finite due to the law of large numbers.

The peculiar sums of type 3) consist of multiplications of indepent terms with zero expectation. Therefore for any n the expectation of these peculiar sums is zero and the limit is also equal to 0.

Consider the peculiar sum $ES^2_{xy,2a}.$  It can be written as $I_1 - I_2$,
$$
I_1=\frac{1}{n}\sum_{k=1}^n \{[\sum_{i=1}^n
(g(x_k-y_i)-J_0)]/\sqrt{n}][\sum_{j=1}^n
g(x_k-y_j)-J_0)]/\sqrt{n}]\},
$$
$$
I_2= \frac {1}{n^2}\sum_{i,j=1}^n(g(x_i-y_j)^2.
$$

Note that $I_2$ tends with $n \to \infty$ to a finite limit due to the law of large numers. And
due to the central limit theorem
under fixed  $X_k$ the random value
$$
\sum_{i=1}^n
[g(x_k-y_i)-J_1]/\sqrt{n}
$$
tends to normal random value with zero expectation and a finite variance.

Others sum of type 2) have a similar behaviour.Thus under $H_0$  there exists a final limit
$$
\lim_{n \to \infty}E(nT_n)^2.
$$
Denote this limit by $d$.

Thus the first part of the lemma is proved.

Let now $H_1$ holds with
$|h|>0$. In this case only the behavior of the following sums
\begin{eqnarray*}
n^2 \{\frac{1}{n^2}\sum_{i,j=1}^n [g(X_i - Y_j)-J_0]\}^2,
\end{eqnarray*}
$$
S_{xy,xx,2}=n^2 (\frac{1}{2n(n-1)n^2})\sum_{k=1}^n \sum_{i,j=1, i\ne j,k, j\ne k}^n[(\ln(1 + (X_k - Y_i)^2)-J_1][(\ln(1 + (X_k - X_j)^2)-J_1],
$$
$$
S_{xy,xx,3}=n^2 (\frac{1}{2n(n-1)n^2})\sum_{i,j=1}^n\sum_{k=1,l=1,l \ne k}^n [(\ln(1 + (X_k - Y_i)^2)-J_1][(\ln(1 + (X_l - X_j)^2)-J_1].
$$
and $S_{xy,yy,2},S_{xy,xy,3}$ that are determined in a similar way that  $S_{xy,xx,2},S_{xy,xx,3}$.
By a direct calculation we will obtain that
$$
\lim_{n \to \infty}E(nT_n)^2=
d + 2J^*(h)J_0 + J^*(h)^2.
$$

Lemma is proved.

\begin{lemma} For $g(x)= x^2$ the following identity holds
$$
\Phi_{nm}= (\bar x - \bar y)^2,
$$
where
$$
\bar x = (\sum_{i=1}^n X_i)/n,
\bar y = (\sum_{i=1}^m Y_i)/m.
$$
\end{lemma}

The proof follows from the known formula [see f.e.\cite{Hoeffding}, p.296]
$$
\frac {1}{n(n-1)}\sum_{1\leq i<j\leq n
} (X_i-X_j)^2=\frac {1}{(n-1)} \sum_{i=1}^n (X_i - \bar x)^2.
$$
by  direct calculations.

Assume that $H_0$ holds. Let  $C$ be an arbitrary positive number,
$$
\tilde{X}=(\tilde{X_{1}},\ldots,\tilde{X_{n}}),\,\,\,
\tilde{Y}=(\tilde{Y_{1}},\ldots, \tilde{Y_{n}}),
$$
where $\tilde{X_{i}}=X_{i}$, if $
|X_{i}| \leq C$ and
 $\tilde{X_i}=C$ if $X_{i}>0$,
  $\tilde{X_i}=-C$ if $X_{i}<0$ otherwise. And $\tilde{Y_{i}}$ are determined similarly. Note that  $0 \leq \ln (1+x^2) \leq x^2$. Therefore there exists a value $t$ that depends from $\tilde{X}$ and $\tilde{Y}$ such that
\begin{equation}
n\{\frac{1}{n^2}\sum_{i,j=1}^n \ln(1 + (\tilde{X_{i}} - \tilde{Y_j})^2 -\frac{1}{n(n-1)}\sum_{i<j} \ln(1 + (\tilde{X_i} - \tilde{X_j})^2) -
\end{equation}
\begin{equation}\label{T_n}
 \frac{1}{n(n-1)}\sum_{i<j}  \ln(1 + (\tilde{Y_i} - \tilde{Y_j})^2)\}=
t
(\sum_{i=1}^n \tilde{X_{i}}/\sqrt{n}-\sum_{i=1}^n \tilde{Y_{i}}/\sqrt{n})^2.
\end{equation}
For constructing the right hand side we applied Lemma 2.
Note  that for distributions $F_1$ and $F_2$  satisfying (\ref{Class}) it follows from Lemma 1 that the variance of the left hand side is finite. Therefore the variance of the right hand side is also finite for arbitrary $C$.
Passing to the limit with $n\to \infty$ we obtain due to the central limit theorem that the right hand side has the limit distribution of the form (9) where  $Z$ has the normal distribution with zero expectation and variance equal to 1. And its variance  is equal to the variance of the left hand side  of $(\ref{T_n})$. Since $C$ is arbitrary we obtain that the limiting distribution has the required form for $H_0$. \\
 For determining $a$ and $b$ in the part (ii) of the theorem  we now can use the equality
 \begin{equation}\label{identity}
 E((aZ+b)^2)^2 =\lim_{n \to \infty}
 E(nT_n)^2,
 \end{equation}
 that follows from  $(\ref{T_n})$.

Since $E Z^2=1, EZ^4=3$,  we have for the left hand side (\ref{identity})
\begin{equation}\label{left}
3a^4 + 6a^2b^2 + b^4.
 \end{equation}
 The formula for the left hand side follows from Lemma 1.
 And the asymptotic behaviour of the power follows from the asymptotic normality of $\sqrt nT_n$.
In order to calculate the right hand side of  (\ref{identity})in (iii)
 the following result is crucial.
\begin{lemma}If $X$ and $Y$ are independent random values with the distribution
$C(0,1)$, then
$$
 E \ln(1+ (X- Y)^2)= \ln 9,\,\,\,\,
 E \ln(1+ (X- Y - \theta)^2)-\ln 9
 =
ln(1+ \theta^2/9).
$$
\end{lemma}
In order to prove this Lemma we need the following integrals
$$
\int_{R}
\frac {\ln(1+(x-y)^2)}{\pi(1+y^2)} dy = \ln(4+x^2),
$$
$$
\quad \int_{R} \frac {\ln(4+x^2)}{\pi(1+x^2)} dx = \ln 9,
$$
(\cite{GradRyzh2007}  4.296.2 and 4.295.7.)
$$
\int_{R} \frac{\ln(4 +(x +\theta)^2 )}{\pi(x^2 +1)} dx = \ln(9+\theta^2),
$$
[see  \cite{PrudBrychMarich1981}, formula (2.6.14.19)].
Using these integrals we obtain
$$
 E \ln(1+ (X- Y - \theta)^2)-\ln 9  = 2\int_{R} \int_{R} \frac{\ln(1+(x-y-\theta)^2)}{\pi^2(1+x^2)(1+y^2)} dx dy -\ln 9
$$
$$
= \int_{R}\frac{\ln(4 +(y+\theta)^2)}{\pi(1+y^2)} dy- \ln 9= \ln(9+\theta^2) -\ln 9 = \ln(1+ \theta^2/9).
$$
Submitting here  $\theta=0$ we obtain both formulas of the Lemma.
Note that $\theta^2=nh^2$ and
$$
\lim_{n\to \infty} n \ln(1+ \theta^2/9)= (1/9)h^2.
$$
Therefore we obtain for the right hand side (8) with some algebra
\begin{equation}\label{right}
3a^4 + \frac{(2\ln 9 )h^2}{9} + \frac{h^4}{81}.
 \end{equation}
 From Lemma 3 and  (\ref{right}) we obtain
$$
b=\frac{1}{3} h,\,\,a^2=\frac{2}{3}\ln 3.
$$
The formula for the power follows from the form of the limiting distribution  (\ref{Distr}).
%% The Appendices part is started with the command \appendix;
%% appendix sections are then done as normal sections
%% \appendix
\bibliographystyle{elsarticle-num-names}
\bibliography{Moscow_testReferences}
\section{References}
 Lehmann E. (1986).  Testing  Statistical  Hypotheses,  Probability  and  Statistics  Series,  Wiley.

Zech,  G. and Aslan, B.(2005).   New test for the multivariate two-sample problem based on the concept of minimum energy.Journal of Statistical Computation and Simulation 75(2), 109–119.

 Wassily Hoeffding, A class of statistics with asymptotically normal distribution.
Ann. Math. Statistics 19 (1948), 293–325.

Buening, H. (2001). Kolmogorov-Smirnov- and Cram`er-von Mises type two-sample tests with various weight functions. Communications in Statistics-
Simulation and Computation, 30, 847-865.

I.S. Gradshteyn and I.M. Ryzhik. Table of Integrals, series and products.Seventh edition AMSTERDAM,BOSTON,HEIDELBERG, LONDON

A. P. Prudnikov, Yu. A. Brychkov, and O. I. Marichev, Integrals and Series. Elementary Functions (Nauka, Moscow, 1981) [in Russian].
\end{document}
%% \section{}
%% \label{}
%% References
%%
%% Following citation commands can be used in the body text:
%% Usage of \cite is as follows:
%%   \cite{key}          ==>>  [#]
%%   \cite[chap. 2]{key} ==>>  [#, chap. 2]
%%   \citet{key}         ==>>  Author [#]
%% References with bibTeX database:
% \bibliographystyle{model1-num-names}
%% New version of the num-names style
\bibliographystyle{elsarticle-num-names}
\bibliography{sample}
%% Authors are advised to submit their bibtex database files. They are
%% requested to list a bibtex style file in the manuscript if they do
%% not want to use model1-num-names.bst.
%% References without bibTeX database:
% \begin{thebibliography}{00}
%% \bibitem must have the following form:
%%   \bibitem{key}...
%%
% \bibitem{}
% \end{thebibliography}
\end{document}
%%
%% End of file `elsarticle-template-1-num.tex'.