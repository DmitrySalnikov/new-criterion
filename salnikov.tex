\documentclass{article}
\usepackage[utf8x]{inputenc}
\usepackage[T2A]{fontenc}
\usepackage[russian]{babel}
\usepackage{indentfirst}
\usepackage{longtable}
\usepackage{amsmath,accents}
\usepackage[left=0.1cm, right=0.1cm, top=1.5cm, bottom=1.5cm]{geometry}

\begin{document}

\begin{eqnarray}
  L_1(Z)&=&\sum_{i,j=1}^{n}{\ln(1+|X_{i}-Y_{j}|)},\\
  L_2(Z)&=&\sum_{i,j=1}^{n}{\ln(1+|X_{i}-Y_{j}|^2)},\\
  L_2^*(Z)&=&\sum_{i,j=1}^{n}{\ln(1+|X_{i}-Y_{j}|^2)}/n^2-\sqrt{B_1 B_2},\\
  B_1 &=& \sum_{i,j=1}^{n}{\ln(1+|X_{i}-X_{j}|^2)}/n(n-1),\\
  B_2 &=& \sum_{i,j=1}^{n}{\ln(1+|Y_{i}-Y_{j}|^2)}/n(n-1),\\
  L_\infty(Z)&=&\sum_{i,j=1}^{n}{\ln(|X_{i}-Y_{j}|)},\\
  LL_{distribution} &=& \text{maximum log likelyhood permutation criterion based on the distribution}
\end{eqnarray}

\begin{longtable}{|c|c|c|c|c|c|}
  \caption{Мощность тестов для распределения Коши без рандомизации, размер выборок n, 1000 итераций, 800 перестановок в каждой итерации}
  \label{table:n50} \\
  \hline
  n & $F_2$ & $L_{2}$ & $LL_{cauchy}$ & wilcox.test & ks.test \\ \hline
  50 & C(1,1) & 79.6 & 85.3 & 74.6 & 79.9 \\
200 & C(0.5,1) & 80.2 & 87.8 & 74.7 & 83.5 \\
1250 & C(0.2,1) & 84.1 & 89.7 & 77.5 & 88.2 \\
5000 & C(0.1,1) & 82.7 & 90.7 & 78.9 & 87.3 \\

  \hline
\end{longtable}

\begin{longtable}{|c|c|c|c|c|}
  \caption{Мощность тестов для распределения Коши без рандомизации, размер выборок $n=1000$, 1000 итераций, 800 перестановок в каждой итерации}
  \label{table:n50} \\
  \hline
  $F_2$ & $L_{2}$ & $LL_{cauchy}$ & wilcox.test & ks.test \\ \hline
  %  &  & LLcauchy & wilcox.test & ks.test \\
C(0, 1) & 4.7 & 5.4 & 4.8 & 6 \\
C(0.05, 1) & 9.2 & 10.6 & 10.2 & 10.7 \\
C(0.1, 1) & 23.1 & 27.8 & 24.8 & 28.7 \\
C(0.15, 1) & 48.7 & 56.8 & 47.5 & 55.7 \\
C(0.2, 1) & 76 & 81.9 & 70.1 & 81.2 \\

  \hline
\end{longtable}

\begin{longtable}{|c|c|c|c|c|c|c|c|c|c|c|}
  \caption{Мощность тестов для распределения Коши без рандомизации, размер выборок $n=200$, 1000 итераций, 800 перестановок в каждой итерации}
  \label{table:n50} \\
  \hline
  $F_2$ & $L_{1}$ & $L_{2}$ & $L_2^*$ & $L_{\infty}$ & $LL_{norm}$ & $LL_{cauchy}$ & $LL_{laplace}$ & $LL_{levy}$ & wilcox.test & ks.test \\ \hline
  %  & L1 & L2 & L2s & Linf & LLnorm & LLcauchy & LLlaplace & LLlevy & wilcox.test & ks.test \\
C(0, 1) & 5.9 & 5.5 & 5.2 & 5.6 & 4.4 & 5.9 & 4.9 & 3.7 & 4.8 & 3.6 \\
\hline
C(0.1, 1) & 8.9 & 7.9 & 5.7 & 8.9 & 4.2 & 9.7 & 4.9 & 3.7 & 8 & 6.7 \\
C(0.2, 1) & 18.3 & 17.9 & 9.2 & 18.1 & 4.2 & 21.8 & 5.1 & 3.6 & 17.6 & 17.4 \\
C(0.3, 1) & 37.6 & 36.2 & 18.4 & 38.1 & 4.2 & 43 & 5.8 & 3.6 & 35 & 38 \\
C(0.4, 1) & 61.5 & 60.1 & 33.8 & 61.4 & 4.2 & 69.3 & 6.1 & 3.5 & 57.1 & 63.2 \\
\hline
C(0, 1.2) & 16.6 & 16.4 & 15.3 & 16.2 & 5 & 19.3 & 6.6 & 5 & 5.1 & 7.4 \\
C(0, 1.4) & 50.4 & 48.7 & 46.6 & 49.9 & 7.6 & 54.5 & 12.4 & 7.3 & 5.4 & 16.7 \\
C(0, 1.6) & 78.3 & 77.1 & 74.8 & 78.6 & 10.5 & 83.6 & 20.5 & 9.2 & 5.4 & 34.8 \\
C(0, 1.8) & 93.6 & 93.4 & 91.6 & 93.7 & 13.6 & 96 & 27.6 & 11.5 & 5.5 & 56.8 \\
\hline
C(0.1, 1.1) & 11.9 & 11.2 & 7.8 & 11.5 & 4.4 & 12.6 & 5.4 & 4.3 & 7.7 & 7.5 \\
C(0.2, 1.2) & 28.4 & 28 & 21 & 28.7 & 5 & 33.4 & 7.2 & 5.2 & 15.1 & 21.5 \\
C(0.3, 1.3) & 56.8 & 54.5 & 43.7 & 55.8 & 6.6 & 60.6 & 9.9 & 6.1 & 27.1 & 40.9 \\
C(0.4, 1.4) & 78.3 & 77.7 & 67.3 & 79.1 & 7.6 & 83.7 & 13.7 & 7.8 & 41.4 & 64.4 \\
\hline
C(0.1, 1.2) & 19.4 & 19 & 16.9 & 19.7 & 5 & 23.2 & 6.9 & 5.1 & 7.8 & 10.3 \\
C(0.2, 1.4) & 58.3 & 56.9 & 50 & 57.3 & 7.6 & 63.1 & 12.5 & 7.7 & 13.8 & 31.3 \\
C(0.3, 1.6) & 87.4 & 85.5 & 80.3 & 86.9 & 10.5 & 90.4 & 20.7 & 9.7 & 22.5 & 58.3 \\
C(0.4, 1.8) & 98.5 & 98.2 & 96.3 & 97.9 & 13.6 & 98.8 & 28.7 & 11.7 & 31.6 & 80.4 \\
\hline

\end{longtable}

\begin{longtable}{|c|c|c|c|c|c|c|c|c|c|c|}
  \caption{Мощность тестов для распределения Коши без рандомизации, размер выборок $n=50$, 1000 итераций, 800 перестановок в каждой итерации}
  \label{table:n50} \\
  \hline
  $F_2$ & $L_{1}$ & $L_{2}$ & $L_2^*$ & $L_{\infty}$ & $LL_{norm}$ & $LL_{cauchy}$ & $LL_{laplace}$ & $LL_{levy}$ & wilcox.test & ks.test \\ \hline
  %  & L1 & L2 & L2s & Linf & LLnorm & LLcauchy & LLlaplace & LLlevy & wilcox.test & ks.test \\
C(0, 1) & 5 & 5 & 4.6 & 5.1 & 5.4 & 4.9 & 5.1 & 5 & 5 & 4.3 \\
\hline
C(0.25, 1) & 10.7 & 10.5 & 6.4 & 10.7 & 5.5 & 11.3 & 5.9 & 4.9 & 10.9 & 10.3 \\
C(0.5, 1) & 28.4 & 26.7 & 14.4 & 27.6 & 5.5 & 30 & 7.5 & 4.9 & 28.9 & 30 \\
C(0.75, 1) & 56 & 53.7 & 32.1 & 55.6 & 5.8 & 61.1 & 10.1 & 5 & 52.2 & 57.1 \\
C(1, 1) & 80.7 & 79.6 & 56.5 & 80.6 & 6.3 & 85.3 & 14.6 & 5.2 & 74.6 & 79.9 \\
\hline
C(0, 1.5) & 22.7 & 21.7 & 21.5 & 22.2 & 11.2 & 22.9 & 13.9 & 7.9 & 5.5 & 9.8 \\
C(0, 2) & 54.7 & 53 & 52.5 & 53.7 & 18.8 & 60.5 & 28.1 & 14 & 6 & 19.6 \\
C(0, 2.5) & 80.6 & 79.8 & 79 & 79.9 & 24.7 & 84.7 & 40.9 & 19.5 & 6.4 & 31.8 \\
C(0, 3) & 92.2 & 92.3 & 91.8 & 91.8 & 32.8 & 95.2 & 52.5 & 26.2 & 6.6 & 49.4 \\
\hline
C(0.25, 1.25) & 15 & 13.9 & 13.1 & 14 & 6.9 & 15.4 & 9 & 5.1 & 9.5 & 10.8 \\
C(0.5, 1.5) & 38.1 & 38.8 & 30 & 36.9 & 11.3 & 42.4 & 15.5 & 7.4 & 20.9 & 27.2 \\
C(0.75, 1.75) & 64.6 & 63.8 & 54.8 & 63.5 & 15.7 & 70.3 & 23.6 & 10.9 & 33.5 & 48.1 \\
C(1, 2) & 83.3 & 82 & 75.7 & 82.2 & 19.2 & 86.9 & 32.4 & 14 & 45.5 & 66.2 \\
\hline
C(0.25, 1.5) & 26.4 & 25.3 & 24.3 & 26.5 & 11.4 & 29 & 14.7 & 7.8 & 9 & 12.8 \\
C(0.5, 2) & 64.6 & 64.6 & 59.5 & 63.2 & 19.1 & 70.9 & 29.1 & 14 & 16.8 & 34.3 \\
C(0.75, 2.5) & 87.2 & 86.2 & 85.5 & 86.2 & 25.2 & 90 & 43.2 & 19.8 & 23.9 & 57.3 \\
C(1, 3) & 96.7 & 96.8 & 95.3 & 96.1 & 33.5 & 97.1 & 55.3 & 24.7 & 30 & 72.3 \\
\hline

\end{longtable}

\begin{longtable}{|c|c|c|c|c|c|c|c|c|c|c|c|}
  \caption{Мощность тестов для Нормального распределения без рандомизации, размер выборок $n=50$, 1000 итераций, 800 перестановок в каждой итерации}
  \label{table:n50} \\
  \hline
  $F_2$ & $L_{1}$ & $L_{2}$ & $L_2^*$ & $L_{\infty}$ & $LL_{norm}$ & $LL_{norm}^{var.eq}$ & $LL_{cauchy}$ & $LL_{laplace}$ & $LL_{levy}$ & wilcox.test & ks.test \\ \hline
  %  & L1 & L2 & Linf & LLnorm & LLcauchy & LLlaplace & LLlevy & wilcox.test & ks.test \\
N(0, 1) & 5.2 & 5.4 & 4.5 & 5.2 & 5.6 & 5.8 & 5 & 5.4 & 4.3 \\
\hline
N(0.25, 1) & 19.4 & 21.8 & 15 & 19.5 & 15.6 & 17.5 & 7.9 & 23.4 & 16.5 \\
N(0.5, 1) & 62.7 & 66.5 & 50.6 & 62.3 & 49.1 & 53.5 & 15.2 & 70 & 53.2 \\
N(0.75, 1) & 93.9 & 95.2 & 86.7 & 93.7 & 83.9 & 89.6 & 26.1 & 96.5 & 90.1 \\
N(1, 1) & 98.9 & 99.1 & 98.4 & 98.9 & 97.5 & 98.5 & 43.6 & 99.4 & 98.7 \\
\hline
N(0, 1.5) & 35.7 & 33.2 & 33.8 & 69.9 & 20.8 & 39.3 & 34.9 & 5.5 & 11.2 \\
N(0, 2) & 89.5 & 89.9 & 84.3 & 99.1 & 68.8 & 91.4 & 75.2 & 5.9 & 37 \\
N(0, 2.5) & 99.3 & 99.4 & 97.7 & 100 & 93.6 & 99.8 & 93.7 & 6.3 & 69 \\
N(0, 3) & 100 & 100 & 100 & 100 & 99.3 & 100 & 98.6 & 6.8 & 89.2 \\
\hline
N(0.25, 1.25) & 24.5 & 25.3 & 20.2 & 38.6 & 19.2 & 26.1 & 8.3 & 18.9 & 16.5 \\
N(0.5, 1.5) & 74.2 & 75.4 & 63.7 & 89.2 & 51.8 & 71.8 & 16.7 & 48.2 & 54.2 \\
N(0.75, 1.75) & 95.9 & 96.3 & 91.8 & 98.9 & 86 & 95.4 & 26.3 & 73 & 84.1 \\
N(1, 2) & 99.5 & 99.6 & 98.2 & 100 & 95.9 & 99.3 & 36.7 & 86.7 & 96 \\
\hline
N(0.25, 1.5) & 47 & 46.5 & 41.2 & 76.1 & 32 & 49.5 & 24.1 & 17.1 & 23.2 \\
N(0.5, 2) & 95 & 95.8 & 91.1 & 99.6 & 82.3 & 95.8 & 56 & 34.7 & 68.1 \\
N(0.75, 2.5) & 99.8 & 99.9 & 98.7 & 100 & 96.8 & 99.8 & 81.2 & 49.5 & 91.1 \\
N(1, 3) & 100 & 100 & 100 & 100 & 99.7 & 100 & 91.6 & 58.1 & 98.2 \\
\hline

\end{longtable}

\begin{longtable}{|c|c|c|c|c|c|c|c|c|c|c|}
  \caption{Мощность тестов для распределения Леви без рандомизации, размер выборок $n=50$, 1000 итераций, 800 перестановок в каждой итерации}
  \label{table:n50} \\
  \hline
  $F_2$ & $L_{1}$ & $L_{2}$ & $L_2^*$ & $L_{\infty}$ & $LL_{norm}$ & $LL_{cauchy}$ & $LL_{laplace}$ & $LL_{levy}$ & wilcox.test & ks.test \\ \hline
  %  & L1 & L2 & L2s & Linf & LLnorm & LLcauchy & LLlaplace & LLlevy & wilcox.test & ks.test \\
Le(0, 1) & 5.5 & 5.7 & 4.8 & 5.8 & 5.6 & 5.8 & 5.6 & 6.2 & 4.6 & 4.7 \\
\hline
Le(0.25, 1) & 7.1 & 6.9 & 5.4 & 13.8 & 5.6 & 6.1 & 5.5 & 65.3 & 10.1 & 8.5 \\
Le(0.5, 1) & 16.2 & 10.8 & 7.5 & 45.9 & 5.6 & 10.7 & 5.5 & 98.9 & 22.7 & 36.8 \\
Le(0.75, 1) & 41 & 27.8 & 9.8 & 81.8 & 5.6 & 23.3 & 5.7 & 100 & 38.7 & 74.5 \\
Le(1, 1) & 69.3 & 53.2 & 17.6 & 94.9 & 5.6 & 43.8 & 5.7 & 100 & 51.9 & 92.3 \\
\hline
Le(0, 1.5) & 13.4 & 12.7 & 9.8 & 15.6 & 6.9 & 14.3 & 6.7 & 22.9 & 20.5 & 13.4 \\
Le(0, 2) & 32.9 & 31.5 & 23.5 & 40.7 & 7.3 & 35 & 8.9 & 57.5 & 48.2 & 37.9 \\
Le(0, 2.5) & 54.3 & 52.3 & 39.7 & 64.3 & 8.5 & 53.4 & 10.4 & 81.9 & 69.5 & 61.1 \\
Le(0, 3) & 71.8 & 69.7 & 54.8 & 81.1 & 9.5 & 70.8 & 12.2 & 93 & 84.3 & 77.9 \\
\hline
Le(0.25, 1.25) & 12.8 & 11.2 & 6.9 & 22.3 & 6.7 & 11.2 & 5.7 & 75.5 & 22.4 & 17.7 \\
Le(0.5, 1.5) & 37.6 & 32.1 & 15.9 & 67.6 & 6.8 & 30.6 & 6.8 & 99.3 & 54.6 & 64 \\
Le(0.75, 1.75) & 72.3 & 64.8 & 35.3 & 92.6 & 6.9 & 52.2 & 7.6 & 100 & 75.1 & 90.2 \\
Le(1, 2) & 90 & 85.4 & 57.8 & 97.9 & 7.3 & 73.4 & 8.9 & 100 & 89.2 & 97.3 \\
\hline
Le(0.25, 1.5) & 21.3 & 19.5 & 11.9 & 35.4 & 6.9 & 20.7 & 6.8 & 82.7 & 37.4 & 30.8 \\
Le(0.5, 2) & 63.7 & 57.4 & 35.5 & 84.7 & 7.3 & 50.2 & 8.9 & 99.6 & 73.8 & 82 \\
Le(0.75, 2.5) & 89.2 & 86.6 & 63.1 & 97.2 & 8.5 & 77.4 & 10.6 & 100 & 92 & 96.3 \\
Le(1, 3) & 97.4 & 96.8 & 84 & 99.5 & 9.5 & 88.9 & 12.6 & 100 & 96.7 & 99.4 \\
\hline

\end{longtable}

\begin{longtable}{|c|c|c|c|c|c|c|c|c|c|c|}
  \caption{Мощность тестов для распределения Лапласа без рандомизации, размер выборок $n=50$, 1000 итераций, 800 перестановок в каждой итерации}
  \label{table:n50} \\
  \hline
  $F_2$ & $L_{1}$ & $L_{2}$ & $L_2^*$ & $L_{\infty}$ & $LL_{norm}$ & $LL_{cauchy}$ & $LL_{laplace}$ & $LL_{levy}$ & wilcox.test & ks.test \\ \hline
  %  & L1 & L2 & L2s & Linf & LLnorm & LLcauchy & LLlaplace & LLlevy & wilcox.test & ks.test \\
La(0, 1) & 5.9 & 5.4 & 6.5 & 5.3 & 5.4 & 5.1 & 6.3 & 5.6 & 4.6 & 4.7 \\
\hline
La(0.25, 1) & 17.6 & 17.1 & 15.5 & 15.4 & 9.5 & 19.2 & 16.9 & 5.5 & 19.5 & 16.3 \\
La(0.5, 1) & 54.6 & 53.9 & 47.8 & 49.5 & 25.9 & 56.6 & 53.4 & 6.8 & 56.7 & 54.2 \\
La(0.75, 1) & 86.9 & 87 & 83.5 & 83.3 & 52.9 & 87.7 & 86.6 & 9.6 & 88.8 & 86.1 \\
La(1, 1) & 97.7 & 98.1 & 96.9 & 96.6 & 79.6 & 97.8 & 97.6 & 13.4 & 97.8 & 97.5 \\
\hline
La(0, 1.5) & 24.3 & 24.6 & 38.2 & 22.6 & 42.4 & 22.3 & 38.5 & 17.4 & 5.7 & 9.2 \\
La(0, 2) & 70.6 & 71.3 & 84.3 & 63.8 & 86.5 & 61.5 & 86.8 & 42.5 & 5.8 & 21.9 \\
La(0, 2.5) & 93.1 & 93.3 & 97.6 & 89.2 & 98.1 & 87.7 & 98.2 & 64 & 6.4 & 40.3 \\
La(0, 3) & 98.6 & 98.7 & 99.5 & 97.5 & 99.7 & 97.5 & 99.7 & 79.3 & 7.1 & 60.4 \\
\hline
La(0.25, 1.25) & 21.8 & 21.5 & 23.8 & 18.8 & 20.5 & 21.7 & 25.4 & 7.6 & 15.9 & 16.1 \\
La(0.5, 1.5) & 59.7 & 58.4 & 63.8 & 54.3 & 55.2 & 57.1 & 67.5 & 13.5 & 39.8 & 45.9 \\
La(0.75, 1.75) & 87.7 & 88.1 & 90.3 & 83.7 & 82.5 & 86.3 & 92.3 & 21.8 & 62.9 & 72.6 \\
La(1, 2) & 97.3 & 97.6 & 98.4 & 95.5 & 95.6 & 96.5 & 98.8 & 29.4 & 78.4 & 91 \\
\hline
La(0.25, 1.5) & 37.4 & 37 & 45.8 & 32.6 & 46.3 & 33 & 47.5 & 16 & 13.9 & 19.3 \\
La(0.5, 2) & 83 & 82.8 & 91.2 & 78.1 & 89.9 & 78.7 & 92.5 & 35.3 & 30 & 50.2 \\
La(0.75, 2.5) & 98.1 & 98.1 & 99.2 & 96.3 & 99.2 & 96.4 & 99.7 & 55 & 45.3 & 76 \\
La(1, 3) & 99.9 & 99.9 & 100 & 99.6 & 100 & 99.8 & 100 & 69.8 & 54.8 & 91.8 \\
\hline

\end{longtable}

\newpage

\begin{longtable}{|c|c|c|c|c|c|c|c|c|c|c|c|}
  \caption{Мощность тестов для распределения Лог-Коши без рандомизации, размер выборок $n=50$, 1000 итераций, 800 перестановок в каждой итерации}
  \label{table:n50} \\
  \hline
  $F_2$ & $L_{1}$ & $L_{2}$ & $L_2^*$ & $L_{\infty}$ & $LL_{norm}$ & $LL_{cauchy}$ & $LL_{laplace}$ & $LL_{levy}$ & $LL_{logcauchy}$ & wilcox.test & ks.test \\ \hline
  %  & L1 & L2 & L2s & Linf & LLnorm & LLcauchy & LLlaplace & LLlevy & LLlogcauchy & wilcox.test & ks.test \\
LC(0, 1) & 6.7 & 6.4 & 6.1 & 6.9 & 4.4 & 6 & 4.6 & 4.5 & 5.8 & 4.8 & 4.5 \\
\hline
LC(0.25, 1) & 9.4 & 9.2 & 6.9 & 8.9 & 4.7 & 12.3 & 4.9 & 5.7 & 11.7 & 8.6 & 9.8 \\
LC(0.5, 1) & 21.4 & 20.6 & 9.6 & 24 & 4.9 & 38.3 & 5.8 & 11.5 & 32.7 & 27.6 & 30 \\
LC(0.75, 1) & 42.4 & 41.6 & 13.9 & 48.2 & 5.8 & 68 & 6.5 & 22.4 & 63.1 & 53 & 58.5 \\
LC(1, 1) & 66.6 & 65.6 & 20.5 & 72.4 & 6.1 & 88 & 7.2 & 38.5 & 85.7 & 75 & 80.7 \\
\hline
LC(0, 1.5) & 15.2 & 14.2 & 12.1 & 17.3 & 8.7 & 11.9 & 9.2 & 7.8 & 24.9 & 5.1 & 9.5 \\
LC(0, 2) & 30.4 & 28.3 & 22.3 & 39.8 & 15.1 & 21.2 & 16.1 & 15.9 & 58.6 & 5.1 & 20.6 \\
LC(0, 2.5) & 45.7 & 42.3 & 30.5 & 61 & 19.9 & 33.6 & 21 & 24.6 & 84.2 & 5.7 & 32.4 \\
LC(0, 3) & 57.8 & 54.7 & 38 & 77.6 & 24.1 & 43.5 & 26 & 35.1 & 92.6 & 5.7 & 47.1 \\
\hline
LC(0.25, 1.25) & 12.9 & 13.3 & 10.7 & 11.3 & 6.8 & 12.9 & 7.1 & 4.9 & 16.1 & 7.9 & 10.6 \\
LC(0.5, 1.5) & 28.1 & 28.2 & 18.8 & 29.4 & 10.2 & 34 & 10.9 & 6.1 & 44.1 & 19.2 & 26.2 \\
LC(0.75, 1.75) & 47.6 & 47.7 & 27.7 & 48.7 & 13.8 & 56.6 & 15.4 & 9.3 & 68.9 & 32.4 & 47.9 \\
LC(1, 2) & 64 & 64.2 & 37.6 & 66.3 & 18.3 & 72.9 & 19.9 & 12 & 86.4 & 44.4 & 67 \\
\hline
LC(0.25, 1.5) & 19.7 & 18.9 & 15.6 & 18.9 & 9.2 & 16.1 & 10.5 & 6 & 28.9 & 7.7 & 12.9 \\
LC(0.5, 2) & 41.8 & 41.3 & 27.9 & 47 & 16.4 & 39.9 & 17.7 & 9.6 & 69.9 & 14.6 & 33.4 \\
LC(0.75, 2.5) & 63.8 & 62.9 & 40.9 & 70.6 & 21.9 & 57.8 & 23.1 & 16.1 & 90.5 & 22.1 & 56.1 \\
LC(1, 3) & 78.4 & 77.4 & 52.5 & 84.8 & 27.2 & 70.8 & 28.1 & 21.7 & 97.1 & 27.4 & 71.9 \\
\hline

\end{longtable}

\end{document}
