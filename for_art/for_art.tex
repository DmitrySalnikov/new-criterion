\documentclass{article}
\usepackage[utf8x]{inputenc}
\usepackage[T2A]{fontenc}
\usepackage[russian]{babel}
\usepackage{indentfirst}
\usepackage{longtable}
\usepackage{amsmath,accents}
\usepackage[left=3cm, right=2cm, top=2cm, bottom=2cm]{geometry}

\begin{document}

$X=(X_1,\ldots,X_n)$, $Y=(Y_1,\ldots,Y_n)$, $Z=(X_1,\ldots,X_n,Y_1,\ldots,Y_n)$.

\begin{eqnarray}
  \label{L0.5}
  L_{0.5}(Z)&=&\sum_{i,j=1}^{n}{\ln(1+|X_{i}-Y_{j}|^{0.5})},\\
  \label{L1}
  L_1(Z)&=&\sum_{i,j=1}^{n}{\ln(1+|X_{i}-Y_{j}|)},\\
  \label{L2}
  L_2(Z)&=&\sum_{i,j=1}^{n}{\ln(1+|X_{i}-Y_{j}|^2)},\\
  \label{LL}
  LL_{distribution} &=& \text{maximum log likelyhood permutation criterion based on the distribution}
\end{eqnarray}

Мощность тестов в процентах. Размер выборок $n=50$. Количество экспериментов $M=1000$ (будет 10000). Количество перестановок $D=800$ (будет 1600). Также думаю, что первый столбец из таблиц можно убрать, написав, что параметры $F_1$ всегда (0, 1).

Также можно при пересчете изменить некоторые параметры распределений, например, во второй строке Table 2 посчитать $N(0, 1)$ vs $N(0.5, 1)$ (сейчас $N(0.6, 1)$).

Рассматривая $L_\gamma$ критерии можно сказать, что чем тяжелее хвост, тем лучше $L_{0.5}$ и наоборот, для нормального $L_2$ лучше.

Из Table 3 убрал $LL_{norm}$ и $LL_{levy}$, так как они хороши только для своих распределений.

\begin{longtable}{|c|c|c|c|c|c|c|c|c|c|}
  \caption{Table 2}
  \label{table:n5} \\
  \hline
  $F_1$ & $F_2$ & $L_{0.5}$ & $L_{1}$ & $L_{2}$ & $LL_{norm}$ & $LL_{cauchy}$ & $LL_{levy}$ & wilcox.test & ks.test \\ \hline
  % c1 & c2 & L05 & L1 & L2 & LLn & LLc & LLl & wilcox.test & ks.test \\
N(0, 1) & N(0, 1) & 4.9 & 5 & 4.3 & 5.3 & 5.1 & 5.2 & 4.1 & 4.1 \\
N(0, 1) & N(0.6, 1) & 71.8 & 75.3 & 78.9 & 75 & 62.2 & 19.2 & 81.7 & 68.4 \\
N(0, 1) & N(0, 2) & 88.6 & 89.3 & 90.3 & 99.1 & 69.8 & 75.5 & 5.5 & 37.7 \\
N(0, 1) & N(0.5, 1.5) & 67.5 & 71 & 72.5 & 88.2 & 51.7 & 16.8 & 45.9 & 50.7 \\
N(0, 1) & N(0.5, 2) & 94.3 & 95.1 & 95.4 & 99.6 & 81.4 & 56.1 & 33.3 & 69.1 \\
\hline
% c1 & c2 & L05 & L1 & L2 & LLn & LLc & LLl & wilcox.test & ks.test \\
C(0, 1) & C(0, 1) & 4.7 & 4.5 & 4.5 & 4.7 & 4.2 & 6.1 & 5.5 & 4.1 \\
C(0, 1) & C(1, 1) & 81.3 & 80.8 & 79.1 & 5.9 & 85.3 & 5.2 & 72.2 & 80.2 \\
C(0, 1) & C(0, 3) & 91.1 & 91.2 & 90.9 & 33.8 & 93.7 & 24.5 & 6.4 & 45.9 \\
C(0, 1) & C(1, 2) & 82.2 & 81.9 & 80.7 & 17.2 & 85.3 & 11.8 & 46.1 & 65.6 \\
C(0, 1) & C(1, 3) & 96.2 & 96.4 & 96.4 & 31 & 97.8 & 21.8 & 28.8 & 70.4 \\
\hline
% c1 & c2 & L05 & L1 & L2 & LLn & LLc & LLl & wilcox.test & ks.test \\
L(0, 1) & L(0, 1) & 5.1 & 5.3 & 5.3 & 5.6 & 5 & 4.3 & 4.6 & 3.4 \\
L(0, 1) & L(1, 1) & 81.2 & 66.1 & 50.9 & 4.4 & 39.6 & 100 & 52.9 & 92.3 \\
L(0, 1) & L(0, 3) & 74.6 & 72.3 & 70.8 & 11.3 & 69.3 & 92.7 & 85.5 & 76.2 \\
L(0, 1) & L(0.5, 1.5) & 43.4 & 34.3 & 29.6 & 5.9 & 27.5 & 98.9 & 50.5 & 58.6 \\
L(0, 1) & L(0.5, 2) & 65.5 & 56.4 & 51.6 & 6.1 & 48.8 & 99.5 & 73.1 & 78 \\
\hline

\end{longtable}

\begin{longtable}{|c|c|c|c|c|c|c|c|c|}
  \caption{Table 3}
  \label{table:n50} \\
  \hline
  $F_1$ & $F_2$ & $L_{0.5}$ & $L_{1}$ & $L_{2}$ & $LL_{cauchy}$ & $LL_{logcauchy}$ & wilcox.test & ks.test \\ \hline
  % c1 & c2 & L05 & L1 & L2 & LLc & LLlc & wilcox.test & ks.test \\
LC(0, 1) & LC(0, 1) & 4.4 & 4.3 & 4.2 & 6 & 6.1 & 6.5 & 5.1 \\
LC(0, 1) & LC(1, 1) & 65.9 & 65 & 64.8 & 87.7 & 86.7 & 71.9 & 80.4 \\
LC(0, 1) & LC(0, 3) & 61.5 & 55.9 & 52.6 & 41.4 & 94.4 & 5.9 & 45.8 \\
LC(0, 1) & LC(1, 2) & 65.1 & 64.5 & 64 & 72.3 & 86 & 44.2 & 66.9 \\
LC(0, 1) & LC(1, 3) & 79.9 & 79.1 & 78.1 & 67.9 & 97.2 & 28.5 & 70.8 \\
\hline

\end{longtable}

\end{document}
